\documentclass[uplatex, a4paper,11pt]{jsarticle}


% 数式
\usepackage{amsmath,amsfonts}
\usepackage{bm}
% 画像
\usepackage[dvipdfmx]{graphicx}
\usepackage[dvipdfmx]{color}
\usepackage[dvipdfmx]{graphicx}
\usepackage{amsmath,amssymb,amsfonts,latexsym}
\usepackage{enumerate}
\usepackage{bm}
\usepackage{xcolor}
\usepackage{graphicx}
\usepackage{ascmac}
\usepackage{setspace}
\usepackage{here}
\usepackage{comment}
\usepackage{bm}
\usepackage{physics}
\usepackage{braket}
\usepackage{fancybox}
\usepackage{fancyhdr}
\usepackage{url}
%\usepackage{jumoline}
\usepackage{listings,jlisting} %日本語のコメントアウトをする場合jlistingが必要
\usepackage{ifthen}
\usepackage{tabularx}
\usepackage{comment}
\usepackage{multirow}

\newlength{\myansboxlength}
\newcommand{\ansbox}[1]{%
\settowidth{\myansboxlength}{#1}%
\setlength{\fboxsep}{2pt}%
\setlength{\fboxrule}{1pt}%
\ifthenelse{\lengthtest{\the\myansboxlength<20pt}}%
{\framebox[1.0cm]{\textgt{\small #1}}}%
{\fbox{\hbox{\hspace{0.35zw}}\textgt{\small #1}\hbox{\hspace{0.35zw}}}}}

\newcommand{\maru}[1]{{\ooalign{%
\hfil\hbox{$\bigcirc$}\hfil\crcr%
\hfil\hbox{#1}\hfil}}}

%\newcommand{\yen}{{\ooalign{Y\crcr\hss=\hss}}}
\def\yen{{\setbox0=\hbox{Y}Y\kern-.85\wd0\vbox{%
\hrule height.1ex width.70\wd0\kern.25ex%
\hrule height.1ex width.70\wd0\kern.40ex}}}

%ここからソースコードの表示に関する設定
%\lstset{
\lstdefinestyle{customc}{
	%プログラム言語(複数の言語に対応,C,C++も可)
 	language = C,
 	%背景色と透過度
 	backgroundcolor={\color[gray]{.95}},
 	%枠外に行った時の自動改行
 	breaklines = true,
 	%自動改行後のインデント量(デフォルトでは20[pt])
 	breakindent = 10pt,
 	%標準の書体
 	%basicstyle = \ttfamily\scriptsize,
  basicstyle=\fontsize{8}{10}\selectfont\ttfamily,
 	%コメントの書体
 	commentstyle = {\itshape \color[cmyk]{1,0.4,1,0}},
 	%関数名等の色の設定
 	classoffset = 0,
 	%キーワード(int, ifなど)の書体
 	keywordstyle = {\bfseries \color[cmyk]{0,1,0,0}},
 	%表示する文字の書体
 	stringstyle = {\ttfamily \color[rgb]{0,0,1}},
 	%枠 "t"は上に線を記載, "T"は上に二重線を記載
	%他オプション:leftline,topline,bottomline,lines,single,shadowbox
 	frame = TBrl,
 	%frameまでの間隔(行番号とプログラムの間)
 	framesep = 5pt,
 	%行番号の位置
 	numbers = left,
	%行番号の間隔
 	stepnumber = 1,
	%行番号の書体
 	numberstyle = \tiny,
	%タブの大きさ
 	tabsize = 4,
 	%キャプションの場所("tb"ならば上下両方に記載)
 	captionpos = t
}
%ここまでソースコードの表示に関する設定
\lstdefinestyle{custombash}{
		rulesepcolor=\color{black},
		language=bash
}
%\newtcblisting{codeBash}[1][]{%
%	title = {Bash},
%	listing options={style=bash,{#1}}
%}
%ここからソースコードの表示に関する設定
\lstdefinestyle{customc_alt}{
  belowcaptionskip=1\baselineskip,
  breaklines=true,
  numbers=left,
  frame=single,
  xleftmargin=\parindent,
  language=C,
  showstringspaces=false,
  basicstyle=\footnotesize\ttfamily,
  keywordstyle=\bfseries\color{green!40!black},
  commentstyle=\itshape\color{purple!40!black},
  identifierstyle=\color{blue},
  stringstyle=\color{orange},
}

\lstdefinestyle{custompy}{
  belowcaptionskip=1\baselineskip,
  breaklines=true,
  numbers=left,
  frame=single,
  xleftmargin=\parindent,
  language=python,
  showstringspaces=false,
  basicstyle=\footnotesize\ttfamily,
  keywordstyle=\bfseries\color{green!40!black},
  commentstyle=\itshape\color{purple!40!black},
  identifierstyle=\color{blue},
  stringstyle=\color{orange},
}

\lstdefinestyle{customasm}{
  belowcaptionskip=1\baselineskip,
  frame=single,
  numbers=left,
  xleftmargin=\parindent,
  language=[x86masm]Assembler,
  basicstyle=\footnotesize\ttfamily,
  commentstyle=\itshape\color{purple!40!black},
}

\lstset{escapechar=@,style=customc}

%ここからソースコードの表示に関する設定
%\lstset{
%  basicstyle={\ttfamily},
%  identifierstyle={\small},
%  commentstyle={\smallitshape},
%  keywordstyle={\small\bfseries},
%  ndkeywordstyle={\small},
%  stringstyle={\small\ttfamily},
%  frame={tb},
%  breaklines=true,
%  columns=[l]{fullflexible},
%  numbers=left,
%  xrightmargin=0zw,
%  xleftmargin=3zw,
%  numberstyle={\scriptsize},
%  stepnumber=1,
%  numbersep=1zw,
%  lineskip=-0.5ex
%}
%ここまでソースコードの表示に関する設定
\makeatletter
\def\ps@plainfoot{%
  \let\@mkboth\@gobbletwo
  \let\@oddhead\@empty
  \def\@oddfoot{\normalfont\hfil-- \thepage\ --\hfil}%
  \let\@evenhead\@empty
  \let\@evenfoot\@oddfoot}
  \let\ps@plain\ps@plainfoot
%\@openrighttrue

%
%\newcommand{\maru}[1]{{\ooalign{%
%\hfil\hbox{$\bigcirc$}\hfil\crcr%
%\hfil\hbox{#1}\hfil}}}
%
\setlength{\textwidth}{\fullwidth}
\setlength{\textheight}{40\baselineskip}
\addtolength{\textheight}{\topskip}
\setlength{\voffset}{-0.55in}
%
%\pagestyle{footnombre}
%
\begin{document}
% START DOCUMENT
%

\title{C言語における論理演算}
\author{S.Matoike}
\date{\today}
\maketitle

\thispagestyle{empty}
\clearpage
\addtocounter{page}{-1}
%\newpage
\setcounter{tocdepth}{1}
\newpage

\lstset{escapechar=@,style=customc}
\lstinputlisting[caption=C言語で論理演算,label=CLogical]{../src/cLogic.c}

\newpage

\lstset{escapechar=@,style=custombash}
\begin{lstlisting}[caption=プログラムの実行結果,label=CLogicalExecuted]
 === NOT:(~a) ===
  a = 1010 0101 1010 0101 1010 0101 1010 0101 = ( 0xA5A5A5A5 )
 ~a = 0101 1010 0101 1010 0101 1010 0101 1010 = ( 0x5A5A5A5A )
 
 === AND:(a & b) ===
 a     = 1010 0101 1010 0101 1010 0101 1010 0101 = ( 0xA5A5A5A5 )
 b     = 0000 0000 0000 0000 0000 0000 1111 1111 = ( 0x000000FF )
 a & b = 0000 0000 0000 0000 0000 0000 1010 0101 = ( 0x000000A5 )
 
 === OR:(a | b) ===
 a     = 1010 0101 1010 0101 1010 0101 1010 0101 = ( 0xA5A5A5A5 )
 b     = 0000 0000 0000 0000 0000 0000 1111 1111 = ( 0x000000FF )
 a | b = 1010 0101 1010 0101 1010 0101 1111 1111 = ( 0xA5A5A5FF )
 
 === XOR:(a ^ b ) ===
 a     = 1010 0101 1010 0101 1010 0101 1010 0101 = ( 0xA5A5A5A5 )
 b     = 1111 1111 1111 1111 1111 1111 1111 1111 = ( 0xFFFFFFFF )
 a ^ b = 0101 1010 0101 1010 0101 1010 0101 1010 = ( 0x5A5A5A5A )
 
 a     = 1010 0101 1010 0101 1010 0101 1010 0101 = ( 0xA5A5A5A5 )
 b     = 0000 0000 0000 0000 0000 0000 0000 0000 = ( 0x00000000 )
 a ^ b = 1010 0101 1010 0101 1010 0101 1010 0101 = ( 0xA5A5A5A5 )
 
 === c = (( a & 3) << 8 ) | (b & 255 ) ===
 a を入力して:   90
 b を入力して: 1355

a                 = 0000 0000 0000 0000 0000 0000 0101 1010 = ( 0x0000005A )
     3            = 0000 0000 0000 0000 0000 0000 0000 0011 = ( 0x00000003 )
a &  3            = 0000 0000 0000 0000 0000 0000 0000 0010 = ( 0x00000002 )
(a & 3) << 8      = 0000 0000 0000 0000 0000 0010 0000 0000 = ( 0x00000200 )

b                 = 0000 0000 0000 0000 0000 0101 0100 1011 = ( 0x0000054B )
    255           = 0000 0000 0000 0000 0000 0000 1111 1111 = ( 0x000000FF )
b & 255           = 0000 0000 0000 0000 0000 0000 0100 1011 = ( 0x0000004B )

c=(a&3)<<8+(b&255)= 0000 0000 0000 0000 0000 0010 0100 1011 = ( 0x0000024B )
\end{lstlisting}

\begin{itemize}
\item 以下のことについて確認します。
\begin{itemize}
  \item 単項演算子のNOTと、二項演算子のAND,OR,XOR
  \item AND演算で、指定したbitのマスキングによる切り出し操作
  \item OR演算で、指定したbitを(無理やり) 1 にしてしまう操作
  \item 0xFF$\cdots$FF とのXOR演算で、bit反転したデータ(これは1の補数)を作れること
  \item 0x00$\cdots$00 とのXOR演算では、何も変わらないこと
  \item 同じデータ同士のXOR演算では、ゼロで埋められたデータになること 
\end{itemize}
\item 応用問題では、a の下 2bit と b の 8bit をつないで、10bit のデータ c を作っています。
\begin{itemize}
\item まず最初に、変数 a の下 2bit を切り出すために、次のマスキング操作を施します\\下 2bit だけが 1 であるような数値 0x00$\cdots$03 との論理積をとります
\item 次に、変数 b の 8bit を納める場所を確保するために、a を左に 8bit シフトしています
\item 最後に、その値と b (これもマスキングによって下 8bit を切り出している) との論理和をとって合計 10bit のデータを作り、それを変数 c に代入しています
\end{itemize}
この様な操作は、10bitのAD変換器の出力を受け取る様な場面でしばしば現れます。\\
\item 【演習】
\begin{itemize}
\item[(1)] 12bitのAD変換器が出力するデータを受け取る場合には、a の下 4bit と b の 8bit をつないで、
12bit のデータ c を作る必要が生じます。さて、どんな操作になるかな。
\item[(2)] a の下6bit と b の下4bit をつないで、10bit のデータ c を作る場合はどうですか。\\
(こんな必要を生じる場面が実際にあるかどうか知らないけど、練習だから) 
\end{itemize}
\item 応用問題の演算
\begin{verbatim}
c = ( (a & 3) << 8 ) | (b & 255)
\end{verbatim}
1bit 左にシフトすると、その値は2倍になる(右に 1bit シフトすると$1/2$)。
8bit 左にシフトしているので、変数 a の値は$2^8=256$倍になりますから、次の様に書いてもOkです。
\begin{verbatim}
c = ( (a & 3) * 256 ) + (b & 255)
\end{verbatim}
%\end{itemize}

%\begin{itemize}
\item 関数main()の冒頭
\begin{verbatim}
  sprintf( frm, "%s%c%s", " = ( 0x%0", CHAR_BIT+'0' ,"X )" );  
\end{verbatim}
この部分では、関数binhex()の中のprintf()で使っている書式文字列frmを作っています。
\begin{verbatim}
  CHAR_BIT + '0'
\end{verbatim}
ここでは、整数のCHAR\_BITをASCII文字のコードに変換しています。\\
数字('0','1','2','3','4','5','6','7','8','9')の各文字は、
文字コード表の上でこの順番に並んでいますから、
数字'0'の文字コードからのオフセットによって、それぞれの文字コードを表す事ができます。
(数字 '8' は 数字 '0' のコードから数値 8 だけ離れた所にある)
%C言語のchar型は 1Byte の整数です
\item 整数変数を2進数の形で出力するためには、printb()のような関数を作らなければなりません。
最初に左端のbit(MSB)だけを1にしたmaskを用意して、
そのmaskの中の1の位置を 1bit ずつ右にシフトしながら後判定反復しています。\\
(1のbitが右端から掃き出されたら、maskは0で「偽」。反復は終了する)
\begin{verbatim}
  do {
        ・・・・・・・
  } while ( mask >>= 1 );   /* mask>>=1 は mask/=2 でもOk */
\end{verbatim}
マスキング操作(mask と v の論理積)の結果によって、0と1の判定をしています。
\begin{verbatim}
  putchar ( mask & v ? '1' : '0' );
\end{verbatim}
この部分を馴染みのある表現で書き直すと、次のようになります。
\begin{verbatim}
  if ( (mask & v) != 0 ) putchar('1');
  else                   putchar('0');
\end{verbatim}
4bit 毎(カウンタ n を 4 で割った余りが0なら)にスペースを1つ出力しています。\\
また、if文の条件判定の「後」で、 n の値をインクリメント $n++$ しています。
\begin{verbatim}
  if ( n++%4==0 ) putchar ( '␣' );
\end{verbatim}
\end{itemize}

\end{document}